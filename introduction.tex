
\section{Contexte}
Dans le courant de l'année 2020, le Professeur Yves Chevallier a réalisé une plateforme web permettant à ses élèves de réaliser des quiz interactifs. Cette plateforme a été réalisée à l'aide du \emph{framework} PHP, Laravel \cite{Laravel} pour le \emph{backend} et du \emph{framework} javascript, VueJs \cite{Vuejs} pour le \emph{frontend}.

Par la suite, un travail de Bachelor a été proposé afin que cette plateforme permette aux élèves d'y effectuer des travaux écrits. Cela apporte des avantages conséquents autant pour les professeurs que pour les élèves. En effet, la plateforme permet une correction partielle du travail écrit (notamment pour les questions QCM) mais elle permet également aux élèves d'écrire du code et de le compiler. Deux choses complétement impossibles avec des tests papiers.


%%if
\section{Situation}
L'application web créée par le Professeur Yves Chevallier est \emph{open-source} et peut être trouvé à l'adresse suivante : \url{https://github.com/heig-vd-tin/heig-quiz}. Voici le dernier \href{https://github.com/heig-vd-tin/heig-quiz/commit/28fb1ac5367931f6aa986041fb992c651c9816cd}{\emph{commit}} au moment du début de mon travail.

L'ajout de travaux écrits a été en partie réalisé, mais cette partie est incomplète. Actuellement, le projet n'est pas complétement fonctionnel et la documentation contient pas mal de lacunes rendant la reprise de ce projet plus difficile que nécessaire. De plus, cette plateforme utilise d'anciennes versions de Laravel et de VueJs. Il convient donc de mettre ces technologies à jour.

\subsection{Environnements de développement}
Comme mentionné précédemment, le code source de ce projet se trouve sur un seul \emph{repository} Github \emph{open-source}. Il s'agit donc d'un \emph{mono-repository}. Cependant la partie \emph{frontend} du projet se trouvant dans le dossier \emph{/resources/js} du projet Laravel, cela rend l'architecture du projet un peu confuse a première vue. Il est donc important de structurer clairement le projet afin de pouvoir le reprendre le plus facilement possible. De plus la documentation présente sur le \emph{repository} est trop peu fournie. Il est donc important de la compléter.

Il est également important de savoir que cette application a été développée dans WSL2 (\emph{Windows Subsystem for Linux v2}). Un utilisateur souhaitant contribuer au projet avec un système d'exploitation Windows se verra donc dans l'obligation d'installer WSL2.
C'est cependant une bonne chose de développer dans l'environnement WSL2 pour les utilisateurs de Windows. En effet, les installations de dépendances y sont bien plus rapides et cela permet une meilleure collaboration avec les utilisateurs d'autres systèmes d'exploitation.

\subsection{Backend}
Le \emph{backend} de cette application est une API REST. Il a été réalisé à l'aide du \emph{framework} PHP, Laravel dans sa version 8. Cette version est désormais quelque peu obsolète et il convient donc de la mettre à jour.

L'API est fonctionnelle, mais présente quelques défauts. Notamment, les codes de retour HTTP ne sont pas toujours conformes aux standards. Par exemple, au lieu d'utiliser le code 201 pour signaler la création d'une ressource, il retourne parfois le code 200. De plus, il y a une incohérence dans l'utilisation des protections pour certaines routes : certaines sont sécurisées par un \emph{middleware}, tandis que d'autres vérifient le rôle de l'utilisateur au début de la fonction. Un dernier détail est que le nom des routes et les méthodes utilisées ne sont pas toujours correcte (Utilisation d'un \emph{POST} au lieu d'un \emph{PATCH}). Pour améliorer ces aspects, il conviendrait de s'assurer que les codes de retour HTTP respectent les standards établis, d'homogénéiser les méthodes de protection des routes, et de veiller aux choix des méthodes employées.

\subsection{Frontend}
Le \emph{frontend} de cette application a été réalisée à l'aide du \emph{framework} javascript, Vue.js dans sa version 2. Il utilise également un peu de Bootstrap et du CSS pour le style de l'application. Comme pour le \emph{backend}, le \emph{frontend} utilise des technologies qui sont désormais vieillissantes et il serait opportun de les mettre à jour.

Comme le \emph{backend}, le \emph{frontend} est fonctionnelle, mais présente un peu plus de défauts. En effet, lors du lancement du \emph{frontend}, on arrive sur une interface qui ne semble n'avoir rien en commun avec notre application. Il nous faut aller chercher dans le router afin de comprendre comment arriver sur l'application. De plus, cette dernière n'a actuellement aucun système de notification après la création de ressource ou lors d'erreur, ce qui rend sont utilisation compliquée et peu intuitive. Un autre aspect négatif est la possibilité pour l'utilisateur de se rendre sur certaines routes auxquelles il n'est pas censé avoir accès. Enfin, certaines fonctionnalités disponibles ne semblent pas fonctionner correctement. Il est donc essentiel de résoudre ces problèmes pour améliorer l'expérience utilisateur.

\subsection{Base de données}
La base de données utilisée par cette application est une base de données MySQL \cite{MySQL}. Cette dernière tourne dans un container  \cite{Docker}, ce qui offre plus de flexibilité aux utilisateurs. En effet, il n'est pas nécessaire d'installer un serveur MySQL sur sa machine pour pouvoir utiliser cette application.

Cette base de données devra être modifiée afin d'y enlever la redondance de certains champs (p. ex. le champ \emph{name} d'un utilisateur peut être calculé en concaténant les champs \emph{firstname} et \emph{lastname}). Il faudra également y ajouter certains attributs et tables afin de pouvoir gérer les travaux écrits ainsi que l'ajout de nouvelles fonctionnalités.

\section{Cahier des charges}
Il me parait maintenant important de définir quels sont les objectifs du projet.

\subsection{Objectifs}
Les objectifs principaux de ce travail de Bachelor s'organisent autour de deux axes :

\begin{enumerate}
    \item Le remaniement du code : comme mentionné précédemment, il est important de tenir le projet et la documentation à jour afin qu'il soit le plus simple possible à reprendre et à modifier. C'est pourquoi je vais me baser sur le code existant afin de réécrire cette plateforme avec les technologies détaillées au chapitre suivant.
    \item L'ajout de l'algorithme SuperMemo d'Anki permettant ainsi aux élèves de se driller avec des quiz de révision ainsi que la finalisation de la gestion des travaux écrits. Pour rappel, l'algorithme SuperMemo permet de noter avec quelle facilité l'étudiant répond à une question. En fonction cette notation, la question reviendra plus ou moins souvent. Cela permet de tomber plus régulièrement sur des questions nous posant un problème. Dans notre cas, nous n'allons pas demander à l'élève de noter la difficulté de la question, mais nous allons nous baser sur le temps que ce dernier a mis à y répondre.
\end{enumerate}

\newpage

Voici les objectifs tels que défini dans le cahier des charges :
\subsection*{Objectifs fonctionnels}
\begin{itemize}
    \item Le projet doit avoir une documentation précise expliquant comment l'installer et le lancer
    \item Un utilisateur doit pouvoir s'identifier à la plateforme à l'aide de son compte de l'école
    \item Un professeur doit pouvoir créer et ajouter des étudiants une classe
    \item Un professeur doit pouvoir créer, via une interface, plusieurs types de questions :
          \subitem – QCM
          \subitem – Texte à trou
          \subitem – Question à développement
          \subitem – Question de code
    \item Les questions utilisent un format Markdown modifié et il doit donc y avoir une page de documentation expliquant comment créer chaque type de question.
    \item Un professeur doit pouvoir créer et gérer un quiz contenant des questions
    \item Un professeur doit pouvoir créer un travail écrit
    \item Un professeur doit pouvoir planifier un travail écrit
    \item Un travail écrit doit s'arrêter après la fin du temps imparti
    \item Un professeur doit pouvoir lancer la correction automatique des questions simples (QCM, texte à trou)
    \item Une fois, le travail écrit corrigé, un professeur doit avoir accès aux statistiques de bonne réponse des questions.
    \item Un élève doit pouvoir faire des quiz.
    \item Un étudiant doit pouvoir utiliser le mode drill du quiz
    \item Un élève doit pouvoir répondre aux questions d'un travail écrit
    \item Un élève doit pouvoir compiler son code
    \item Un élève doit pouvoir rendre son examen avant la fin du temps imparti
\end{itemize}

\subsection*{Objectifs non-fonctionnels}

\begin{itemize}
    \item L'interface doit être fluide et intuitive pour les utilisateurs
    \item L'application doit être fonctionnelle et fiable
    \item Un CI/CD doit être mis en place afin de faciliter la reprise du projet et son déploiement
    \item L'application est open source et le code est hébergé sur GitHub
    \item Les réponses d'un élève ne doivent pas pouvoir être modifiée après la fin de l'examen
    \item Les questions de l'examen ne doivent pas être modifiables après l'examen en question
    \item Les messages d'erreur doivent être clairs et compréhensibles pour l'utilisateur
\end{itemize}
%%fi
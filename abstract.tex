% Francais
Dans le courant de l'année 2020, le professeur Yves Chevallier a réalisé une plateforme web permettant à ses élèves de réaliser des quiz interactifs. Cette plateforme a été réalisée à l'aide du framework PHP, Laravel pour le backend et du framework javascript, Vue.js pour le frontend.

Un précédent travail de Bachelor a eu pour but d'y ajouter la gestion de travaux écrits. La gestion de ces derniers n'étant pas entièrement fonctionnelle, un second travail de Bachelor a été soumis.

Dans un premier temps, ce travail apportera un refactoring du code pour porter le projet sur la version 10 de Laravel et sur la version 3 de Vue.js.

Dans un second temps, il complétera la gestion des travaux écrits comportant différents types de questions telles que des Questions à choix multiples, des questions de développement, des textes à trou, des questions de code, etc.

Finalement il aura pour but d'implémenter l'algorithme SM-2 (SuperMemo - Wikipedia) de Anki afin de permettre aux étudiants un drill avec des questions générées.

Les utilisateurs ciblés par cette plateforme sont les professeurs et étudiants de la HEIG-VD, c'est pourquoi l'authentification sera effectuée au travers du keycloak de l'école.

\asterism

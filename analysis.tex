Le but de cette section est d'analsyer quels sont les besions des différents utilisateurs et de trouver quelle est la meilleure manière d'y répondre. Un autre point très important est le détails des choix technologiques. Je vais donc tous les passer en revue et expliquer pourquoi j'ai choisi cette technologie.
\section{Besoins}

Dans un premier temps, il convient d'identifier quels seront les différents types d'utilisateurs de cette plateforme de quiz. Il y a selon moi deux types bien distinct d'utilisateurs :
\begin{itemize}
    \item Les étudiants
    \item Les professeurs
\end{itemize}

Les premiers utilisent cette plateforme afin de répondre à des quizs. La forme de ces derniers peuvent varier entre un simple quiz pour valider un devoir, un drill dans le but de réviser un examen ou finalement l'examen en lui-même. Ils ont donc besoin d'une interface où ils peuvent voir et choisir un quiz parmi tous ceux dont l'accès leurs est autorisé. Il doit pouvoir répondre au quiz et dans le cadre d'un examen ou d'un devoir il doit pouvoir le rendre. Il doit également pouvoir naviguer dans le quiz.

Les professeurs quant à eux ont des besoins bien différents. Ils veulent principalement créer des quizs avec des questions de plusieurs types tels que un QCM, des textes à trous ou encore des questions de code. Ils doivent également pouvoir regrouper leurs étudiants en différentes classes et autoriser cette classe à répondre à certains quizs. Dernièrement, ils ont besoin de pouvoir corriger automatiquement certaines questions comme les QCM.



\section{Technologies}
